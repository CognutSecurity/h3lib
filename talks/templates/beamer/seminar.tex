\documentclass{beamer}
\usepackage[german,english]{babel}

%\usecolortheme{squid}
%\usecolortheme{elephant}
%\usecolortheme{fish}

%\setbeamertemplate{headline}[ias]%[english]

\mode<beamer>
{
  \AtBeginSection[]
  {
    \begin{frame}<beamer>
      \frametitle{Outline}
      \tableofcontents[currentsection,currentsubsection]
    \end{frame}
  }
}

%% General Information

\title[Short Paper Title] % (optional, use only with long paper titles)
{Presentation Title}

\subtitle{Presentation Subtitle} % (optional)

\author[Author, Another] % (optional, use only with lots of authors)
{F.~Author\inst{1} \and S.~Another\inst{2}}
% - Use the \inst{?} command only if the authors have different
%   affiliation.

\institute[Universities of Somewhere and Elsewhere] % (optional, but mostly needed)
{
  \inst{1}%
  Department of Computer Science\\
  University of Somewhere
  \and
  \inst{2}%
  Department of Theoretical Philosophy\\
  University of Elsewhere}
% - Use the \inst command only if there are several affiliations.
% - Keep it simple, no one is interested in your street address.

\date[Short Occasion] % (optional)
{Date / Occasion}

\subject{Talks}
% This is only inserted into the PDF information catalog. Can be left
% out.

\newcommand{\texcmd}[1]{{\ttfamily\textbackslash #1}}

%% Dokument

\begin{document}

\maketitle

\begin{frame}
  \frametitle{Outline}
  \tableofcontents
  % You might wish to add the option [pausesections]
\end{frame}

\section{Tests}

\subsection{First Subsection}

\begin{frame}
  \frametitle{Test}
  \begin{minipage}{0.45\textwidth}
    \begin{block}{linke Seite}
      \vspace{3cm}
    \end{block}
  \end{minipage}
  \hfill
  \begin{minipage}{0.45\textwidth}
    \begin{block}{rechte Seite}
      \vspace{4cm}
    \end{block}
  \end{minipage}
\end{frame}

\begin{frame}
  \frametitle{Make Titles Informative.}

  You can create overlays\dots
  \begin{itemize}
  \item using the \texttt{pause} command:
    \begin{itemize}
    \item
      First item.
      \pause
    \item
      Second item.
      \begin{itemize}
        \item first subitem
        \item second subitem
      \end{itemize}
    \end{itemize}
  \item
    using overlay specifications:
    \begin{itemize}
    \item<3->
      First item.
    \item<4->
      Second item.
    \end{itemize}
  \end{itemize}
\end{frame}


\subsection{Second Subsection}

\begin{frame}
  \frametitle{Make Titles Informative.}

  \begin{block}{Block}
    mehr Text
  \end{block}

  \begin{alertblock}{Alert Block}
    mehr Text
  \end{alertblock}

  \begin{example}
    mehr Text
  \end{example}
\end{frame}

\begin{frame}
  \frametitle{Make Titles Informative.}

  \begin{enumerate}
    \item eins
      \begin{enumerate}
        \item one
          \begin{enumerate}
            \item un
            \item deux
          \end{enumerate}
        \item two
      \end{enumerate}
    \item zwei
  \end{enumerate}
\end{frame}



\section{Summary}

\begin{frame}
  \frametitle<presentation>{Summary}

  % Keep the summary *very short*.
  \begin{itemize}
  \item
    The \alert{first main message} of your talk in one or two lines.
  \item
    The \alert{second main message} of your talk in one or two lines.
  \item
    Perhaps a \alert{third message}, but not more than that.
  \end{itemize}

  % The following outlook is optional.
  \vskip0pt plus.5fill
  \begin{itemize}
  \item
    Outlook
    \begin{itemize}
    \item
      Something you haven't solved.
    \item
      Something else you haven't solved.
    \end{itemize}
  \end{itemize}
\end{frame}

\end{document}

Making Slides:
pdflatex seminar.tex

Making Handout:
beamer-handout --frame true seminar.tex

